\documentclass[12pt, twoside]{article}
\usepackage{jmlda}
\newcommand{\hdir}{.}

% Здесь можно определять собственные команды, они будут действовать только внутри статьи:
\newenvironment{coderes}%
    {\medskip\tabcolsep=0pt\begin{tabular}{>{\small}l@{\quad}|@{\quad}l}}%
    {\end{tabular}\medskip}

\begin{document}

\title{Ранее прогнозирование достаточного объема выборки для обобщенно линейной модели}
\author{Жолобов В. А. Малиновский Г. Стрижов В. В.}
%\email{info@jmlda.org}
%\organization{ФИЦ <<Информатика и управление>> РАН, г.~Москва, ул.~Вавилова, 44/2}
\abstract{Исследуется проблема планирования эксперимента. Задача ранего прогнозирования важна в медицинском применении, особенно в случаях дорогостоящих измерений иммунных биомаркеров. Решается задача оценивания достаточного объема выборки по данным. Предполагается, что выборка является простой. Она описывается адеватной моделью. Иначе, выборка порождается фиксированной вероятностной моделью из известного класса моделей. Объем выборки считается достаточным, если модель восстанавливается с достаточной достоверностью. Исследуется зависимость функции ошибки от объема данных. Исследуется зависимость модели от редуцированной матрицы ковариации параметров $GLM$. Требуется, зная модель, оценить достаточный объем выборки на ранних этапах сбора данных. Созданы алгоритмы определения достаточного объема данных на раннем этапе. Проведен вычислительный эксперимент с использованием синтетических данных.}
\titleEng{Style guide for authors}
\authorEng{JMLDA editorial board}
\organizationEng{Federal Research Center ``Computer Science and Control'' of RAS, 44/2~Vavilova~st., Moscow, Russia}
\abstractEng{
    This document explains how to prepare papers using \LaTeXe\ typesetting system and \texttt{jmlda.sty} package.
}
%\doi{10.21469/22233792}
%\receivedRus{01.01.2017}
%\receivedEng{January 01, 2017}

\maketitle
%\linenumbers
\section{Введение}
%связана с планированием эксперимента.
%Как снизить стоимость данных для исследований в несколько раз?
%Каким образом предсказать ее минимально необходимый объем по небольшому числу измерений?
%, порядка трех тысяч евро
Работа посвящена задаче оценивания достаточного объема выборки на раннем этапе сбора данных. Задача возникла из условия, когда необходимо провести крупное исследование, а сбор данных стоит больших денег. Для примера можно взять медицинское исследование, такой как анализ крови. Существуют такие виды анализа крови, которые стоят достаточно приличных денег для людей. Для того, чтобы снизить стоимость данных для исследований в несколько раз необходимо построить модель, а для модели нужно собрать выборку. Поэтому в данной работе рассматривается задача построения алгоритма для предсказания оптимального набора данных при заданной модели. Предлагаемый в данной работе метод должен на малой выборке спрогнозировать ошибку на пополняемой большой. Выборка считается простой, то есть удовлеторяет простому распределению. Предлагается использовать два разных метода: полного перебора и генетический алгоритм
 %При планировании эксперимента требуется оценить минимальный объем данных - количество производимых измерений некоторого набора параметров. 
  
Кроме этих методов ранее задача прогнозирования достаточного объема выборкы решалась в работе~\cite{oai:dialnet.unirioja.es:ART0000605621}. Здесь был предложен метод, основанный на технике кросс-валидации и расстоянии Кульбака-Лейблера между двумя распределениями параметров модели, оцениваемых на аналогичных подмножествах данных. Похожая задача информационного поиска решалась в работах~\cite{journals/eswa/KulunchakovS17, oai:HAL:hal-01118844v1}. Здесь для создания простых структурированных функций информационного поиска используется модернизированный генетический алгоритм. Модернизированность генетического алгоритма заключается в том, что он способен бороться со стагнацией признаков.

В данной работе используются два метода. Основной из них~--- это метод полного перебора. Необходимо подобрать такую функцию, которая является монотонной и достаточно гладкой, то есть гарантируется непрерывная дифференцируемость до второго порядка. Метод заключается в том, что он аппроксимирует зависимость функции ошибки от объема данных по малому объему выборки, чтобы с достаточной точностью предсказывать поведение функции ошибки. Считается, что модель в этой задаче задана и зависит от редуцированной матрицы ковариации параметров $GLM$. Также предложен способ генерации такой функции через генетический алгоритм.



Вычислительный эксперимент проводится на синтетических данных $Boston\ Housing$ и $Diabets$. Вначале реализуем метод полного перебора. Разделяем выборку на два непересекающихся множества. Строим два графика поверхности выборок: первую получаем с помощью бутстрепа~\cite{Bishop06} для подвыборки фиксированного объема, вторую через аппроксимацию. Чтобы получить аппроксимирующую поверхность, решается оптимизационную задачу. Затем повторяем действия, используя уже для поиска аппроксимирующей функции генетический алгоритм. Решение этой задачи позволит находить оптимальное значение объема выборки. 



\bibliographystyle{unsrt}
\bibliography{Cites}

\end{document}
